\documentclass[]{vvsu}

\vvsuyear{2025}

%%%%%%%%%%%%%%%%%%%

\usepackage{graphicx} % для изображений
\usepackage{tabularray} % для таблиц
\usepackage{siunitx} % для обозначений (процент, градус)
\usepackage{listings} % для листингов кода

% Список путей, где будут искаться изображения и файлы
\graphicspath{{images/}}

% Автор документа
\author{А.В. Джаманшалов}

% Настройка стилей для листингов кода
\input{listing_styles.tex}

%%%%%%%%%%%%%%%%%%%

\begin{document}

% Шапка
\vvsuhead{\linespread{1}\selectfont{}МИНОБРНАУКИ РОССИИ\\
\vspace{10pt}Федеральное государственное бюджетное образовательное учреждение\\
высшего образования\\
\fontsize{13}{13}\selectfont{}<<ВЛАДИВОСТОКСКИЙ ГОСУДАРСТВЕННЫЙ УНИВЕРСИТЕТ>>\\
(ФГБОУ ВО <<ВВГУ>>)\\
\vspace{10pt}\fontsize{12}{12}\selectfont{}ИНСТИТУТ ИНФОРМАЦИОННЫХ ТЕХНОЛОГИЙ И АНАЛИЗА ДАННЫХ\\
КАФЕДРА ИНФОРМАЦИОННЫХ ТЕХНОЛОГИЙ И СИСТЕМ}

% Название отчета
\title{Отчет\\по лабораторной работе №4}
\subtitle{по дисциплине\\<<Информатика и программирование>>}

% Участники работы
\member{Студент\\ гр. БИН-25-2}{А.В. Джаманшалов}
\member{Ассистент\\ преподавателя}{М.В. Водяницкий}

% Вывод титульника
\maketitle

% Задание
\begin{addition}{Задание}
  Выполнить задания и оформить отчет по стандартам ВВГУ.

  \textit{\textbf{Задание 1.}}  
  Написать программу, которая определяет, как будет вести себя кондиционер. Если температура в помещении 20 градусов и выше, то кондиционер выключается, если меньше - включается. Температура должна вводится пользователем с консоли.

  Пример:\\
    Введите температуру: 18\\
    Кондиционер включен

  \textit{\textbf{Задание 2.}}  
  Год делится на четыре сезона: зима, весна, лето и осень. Написать программу, которая запрашивает у пользователя номер месяца и выводит к какому сезону этот месяц относится.

  Пример:\\
    Введите номер месяца: 4\\
    Это весна 

  \textit{\textbf{Задание 3.}}  
  Считается, что один год, прожитый собакой, эквивалентен семи человеческим годам. При этом зачастую не учитывается, что собаки становятся абсолютно взрослыми уже к двум годам. Таким образом, многие предпочитают каждый из первых двух лет жизни собаки приравнивать к 10.5 годам человеческой жизни, а все последующие к 4.
  
  Написать программу, которая будет переводить собачий возраст в человеческий. Программа должна корректно обрабатывать входные данные и выводить соответствующие сообщения об ошибках:

  \begin{vvsu_itemize}
    \item Если вводится не число
    \item Если вводится число меньше 1
    \item Если вводится число большее 22
  \end{vvsu_itemize}

  Пример:\\
    Введите возраст собаки (в годах): 5\\
    Возраст собаки в человеческих годах: 33.0

  Пример:\\
    Введите возраст собаки (в годах): 0\\
    Ошибка: возраст должен быть не меньше 1

  \textit{\textbf{Задание 4.}}  
  Число делиться на 6 только в случае соблюдения двух условий:
  \begin{vvsu_itemize}
    \item Последняя цифра четная
    \item Сумма всех цифр делиться на 3
  \end{vvsu_itemize}
  Написать программу, которая выведет делиться ли введенное число на 6 или нет.

  \textit{\textbf{Задание 5.}}  
  Написать программу, которая будет проверять пароль на надежность. Пароль считается надежным, если его длина не менее 8 символов и если он содержит:

  \begin{vvsu_itemize}
    \item Заглавные буквы латиницы
    \item Строчные буквы латиницы
    \item Числа
    \item Специальные знаки
  \end{vvsu_itemize}

  В случае, если пароль не проходит по одному из условий, необходимо сообщить пользователю каким именно условиям он не удовлетворяет.

  Пример:\\
    Введите пароль: qwerty\\
    Пароль ненадежный: отсутствуют заглавные буквы, числа и специальные символы

  \textit{\textbf{Задание 6.}}  
  Написать программу, которая определяет, является ли введенный пользователем год високосным. Год считается високосным, если он делится на 4, но не делится на 100, либо если он делится на 400.

  Пример:\\
    Введите год: 2024\\
    2024 - високосный год

  \textit{\textbf{Задание 7.}}  
  Написать программу, которая запрашивает у пользователя три числа и выводит на экран наименьшее из них. При решении нельзя использовать встроенные функции min() и max().

  Пример:\\
    Введите три числа: 8 3 5\\
    Наименьшее число: 3

  \textit{\textbf{Задание 8.}}  
  В магазине проводится акция. Акция работает по следующим правилам:

  \begin{vvsu_itemize}
    \item Сумма < 1000 => скидка - 0\%
    \item Сумма < 5000 => скидка - 5\%
    \item Сумма < 10000 => скидка - 10\%
    \item Сумма > 10000 => скидка - 15\%
  \end{vvsu_itemize}

  Напишите программу, которая запрашивает сумму покупки и выводит размер скидки и итоговую сумму к оплате.
  
  Пример:\\
    Введите сумму покупки: 7500\\
    Ваша скидка: 10%\\
    К оплате : 6750.0
  
  \textit{\textbf{Задание 9.}}  
  Написать программу, которая определяет время суток по введенному часу (целое число от 0 до 23).

  \begin{vvsu_itemize}
    \item С 0 до 5 часов - ночь
    \item С 6 до 11 часов - утро
    \item С 12 до 17 часов - день
    \item С 18 до 23 часов - вечер
  \end{vvsu_itemize}
  
  Пример:\\
    Введите час (0–23): 20\\
    Сейчас вечер

  \textit{\textbf{Задание 10.}}  
  Написать программу, которая определяет, является ли введенное число простым. Число называется простым, если оно больше 1 и делится только на 1 и само себя. Программа должна корректно обрабатывать некорректный ввод и выводить соответствующие сообщения об ошибках.
  
  Пример:\\
    Введите число: 17\\
    17 - простое число
\end{addition}


% Содержание
\toc

% Глава - Выполнение работы
\section{Выполнение работы}

% Подглава - Задание 1
\subsection{Задание 1}

Программа запрашивает у пользователя температуру и в зависимости от введенного значения решает, нужно ли включать кондиционер. Если температура равна или выше 20 градусов, программа сообщает, что кондиционер выключен. Если же температура ниже 20 градусов, выводится сообщение о том, что кондиционер включен. На рисунке \ref{fig:code_task_1} представлен код программы.

\begin{vvsu_figure}{Листинг программы для задания 1}{fig:code_task_1}
  \begin{minipage}{.75\textwidth}
    \lstinputlisting[language=Python,basicstyle=\fontsize{10}{10}\linespread{1}\selectfont\ttfamily]{code/task1.py}
  \end{minipage}
\end{vvsu_figure}

% Подглава - Задание 2
\subsection{Задание 2}

 Эта программа определяет время года по номеру месяца.Пользователь вводит число от 1 до 12, и программа сообщает, какое сейчас время года.Если введен номер 12, 1 или 2, программа выводит "Сейчас зима". При вводе чисел от 3 до 5 появляется сообщение "Сейчас весна". Для месяцев с 6 по 8 программа покажет "Сейчас лето", а для период с сентября по ноябрь (9-11) — "Сейчас осень".. На рисунке \ref{fig:code_task_2} представлен код программы.

\begin{vvsu_figure}{Листинг программы для задания 2}{fig:code_task_2}
  \begin{minipage}{.75\textwidth}
    \lstinputlisting[language=Python,basicstyle=\fontsize{10}{10}\linespread{1}\selectfont\ttfamily]{code/task2.py}
  \end{minipage}
\end{vvsu_figure}

% Подглава - Задание 3
\subsection{Задание 3}

Программа проверяет, чтобы возраст собаки был от 1 до 22 лет. Если введено число меньше 1 или больше 22, выводится сообщение об ошибке.Для расчета используется два подхода: если собаке 2 года или меньше, каждый собачий год равен 10.5 человеческим. Если собаке больше 2 лет, то первые два года считаются как 21 человеческий год (2 × 10.5), а каждый последующий год равен 4 человеческим годам. На рисунке \ref{fig:code_task_3} представлен код программы.

\begin{vvsu_figure}{Листинг программы для задания 3}{fig:code_task_3}
  \begin{minipage}{.75\textwidth}
    \lstinputlisting[language=Python,basicstyle=\fontsize{10}{10}\linespread{1}\selectfont\ttfamily]{code/task3.py}
  \end{minipage}
\end{vvsu_figure}

% Подглава - Задание 4
\subsection{Задание 4}

Эта программа проверяет, делится ли введенное число на 6. Пользователь вводит любое целое число, и программа анализирует его по двум признакам.Чтобы число делилось на 6, оно должно делиться одновременно на 2 и на 3. Программа проверяет четность числа через остаток от деления на 2, а для проверки делимости на 3 вычисляет сумму цифр числа. Если сумма цифр делится на 3, то и само число делится на 3.Если оба условия выполняются — число четное и сумма его цифр делится на 3 — программа сообщает, что число делится на 6. В противном случае выводится сообщение, что число не делится на 6. На рисунке \ref{fig:code_task_4} представлен код решения. На рисунке \ref{fig:code_task_4} представлен код решения.

\begin{vvsu_figure}{Листинг программы для задания 4}{fig:code_task_4}
  \begin{minipage}{.75\textwidth}
    \lstinputlisting[language=Python,basicstyle=\fontsize{10}{10}\linespread{1}\selectfont\ttfamily]{code/task4.py}
  \end{minipage}
\end{vvsu_figure}


% Подглава - Задание 5
\subsection{Задание 5}

Эта программа проверяет надежность пароля. Пользователь вводит пароль, и программа анализирует его по нескольким критериям безопасности.Программа проверяет пять основных требований к надежному паролю: длина не менее 8 символов, наличие заглавных букв, наличие строчных букв, присутствие цифр и обязательное использование специальных символов (знаков препинания, математических операторов и других небуквенно-цифровых символов).Если пароль не соответствует какому-либо из этих требований, программа собирает все найденные ошибки в список и выводит их перечислением. Если все требования выполнены, программа сообщает, что пароль надежный. На рисунке \ref{fig:code_task_5} представлен код программы.

\begin{vvsu_figure}{Листинг программы для задания 5}{fig:code_task_5}
  \begin{minipage}{.75\textwidth}
    \lstinputlisting[language=Python,basicstyle=\fontsize{10}{10}\linespread{1}\selectfont\ttfamily]{code/task5.py}
  \end{minipage}
\end{vvsu_figure}

% Подглава - Задание 6
\subsection{Задание 6}

Эта программа определяет, является ли год високосным. Пользователь вводит любой год, и программа проверяет его по правилам високосных лет.Год считается високосным в двух случаях: если он делится на 4, но не делится на 100, либо если он делится на 400. Программа проверяет эти условия и выводит соответствующее сообщение.. На рисунке \ref{fig:code_task_6} представлен код программы.

\begin{vvsu_figure}{Листинг программы для задания 6}{fig:code_task_6}
  \begin{minipage}{.75\textwidth}
    \lstinputlisting[language=Python,basicstyle=\fontsize{10}{10}\linespread{1}\selectfont\ttfamily]{code/task6.py}
  \end{minipage}
\end{vvsu_figure}

% Подглава - Задание 7
\subsection{Задание 7}

Эта программа находит наименьшее из трех чисел. Пользователь вводит три числа через пробел, и программа определяет, какое из них самое маленькое.Программа последовательно сравнивает числа между собой. Сначала проверяется, является ли первое число меньшим или равным двум другим. Если это так, выводится первое число. Если нет, проверяется второе число — меньше ли оно или равно первому и третьему. Если и это не выполняется, значит третье число является наименьшим, и оно выводится как результат. На рисунке \ref{fig:code_task_7} представлен код программы.

\begin{vvsu_figure}{Листинг программы для задания 7}{fig:code_task_7}
  \begin{minipage}{.75\textwidth}
    \lstinputlisting[language=Python,basicstyle=\fontsize{10}{10}\linespread{1}\selectfont\ttfamily]{code/task7.py}
  \end{minipage}
\end{vvsu_figure}

% Подглава - Задание 8
\subsection{Задание 8}

Эта программа рассчитывает скидку и итоговую сумму покупки. Пользователь вводит общую сумму покупки, и программа автоматически определяет размер скидки в зависимости от потраченной суммы.Скидка начисляется по прогрессивной шкале: при сумме до 1000 рублей скидки нет, от 1000 до 5000 рублей дается 5 скидки, от 5000 до 10000 рублей — 10 скидки, а при покупках на 10000 рублей и больше — максимальная скидка 15.После расчета скидки программа вычисляет окончательную сумму к оплате и показывает пользователю как размер полученной скидки в процентах, так и итоговую сумму для оплаты. На рисунке \ref{fig:code_task_8} представлен код программы.

\begin{vvsu_figure}{Листинг программы для задания 8}{fig:code_task_8}
  \begin{minipage}{.75\textwidth}
    \lstinputlisting[language=Python,basicstyle=\fontsize{10}{10}\linespread{1}\selectfont\ttfamily]{code/task8.py}
  \end{minipage}
\end{vvsu_figure}

% Подглава - Задание 9
\subsection{Задание 9}

Эта программа определяет время суток по введенному часу. Пользователь вводит число от 0 до 23, представляющее час, и программа сообщает, какое сейчас время суток.Программа разделяет сутки на четыре периода: ночь с 0 до 5 часов, утро с 6 до 11 часов, день с 12 до 17 часов и вечер с 18 до 23 часов. Каждому периоду соответствует свое сообщение. На рисунке \ref{fig:code_task_9} представлен код программы.

\begin{vvsu_figure}{Листинг программы для задания 9}{fig:code_task_9}
  \begin{minipage}{.75\textwidth}
    \lstinputlisting[language=Python,basicstyle=\fontsize{10}{10}\linespread{1}\selectfont\ttfamily]{code/task9.py}
  \end{minipage}
\end{vvsu_figure}

% Подглава - Задание 10
\subsection{Задание 10}

Эта программа проверяет, является ли число простым. Пользователь вводит целое число, и программа анализирует его свойства.Программа сначала проверяет особые случаи: отрицательные числа считаются ошибкой, а числа 0 и 1 не относятся ни к простым, ни к составным. Для чисел больше 1 программа проверяет делимость на все числа от 2 до самого числа (не включая его). Если находится хотя бы один делитель, число считается составным. Если делителей нет, число является простым. На рисунке \ref{fig:code_task_10} представлен код программы.

\begin{vvsu_figure}{Листинг программы для задания 10}{fig:code_task_10}
  \begin{minipage}{.75\textwidth}
    \lstinputlisting[language=Python,basicstyle=\fontsize{10}{10}\linespread{1}\selectfont\ttfamily]{code/task10.py}
  \end{minipage}
\end{vvsu_figure}

Спасибо за внимание !

\end{document}